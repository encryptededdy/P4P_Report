\documentclass[10pt]{article}
\usepackage{geometry}
\usepackage{setspace}
\usepackage{times}
\usepackage{titlesec}
\usepackage{enumitem}
\usepackage{graphicx}

\titleformat{\section}{\normalfont\bfseries\filcenter}{\thesection. }{0.8em}{}
\titlespacing*{\section}{0pt}{2.0ex}{-1ex}

\titleformat{\subsection}{\normalfont\bfseries}{\thesubsection. }{0em}{}
\titlespacing*{\subsection}{0pt}{2.0ex}{-1ex}

\titleformat{\subsubsection}{\normalfont\it}{\thesubsubsection. }{1.0em}{}
\titlespacing*{\subsubsection}{0pt}{2.0ex}{-1ex}

\geometry{
  a4paper,
  total={165mm,226mm},
  left=20mm,
  right=20mm,
  top=24.5mm,
  textwidth=165mm,
  textheight=235mm
}

\linespread{1.85}

\begin{document}

\noindent \textbf{ABSTRACT:} TODO!

\section{Introduction}
A solid understanding of Data Structures and Algorithms (DSA) is an important skill for any student in the field of Computer Science, since they underpin many of the fundamentals of Computer Science. However, as is the case with many conceptually demanding topics, students can find learning DSA challenging early on in their studies\cite{7600449} due to their inability to correlate DSA concepts with real-world objects and problems.\par
Educational tools have often been proposed that either take advantage of Game-based Learning (GBL) or Algorithm Visualisation (AV) in order to help students better learn concepts in computer science.\par
Game-based Learning is the use of games with traditional game elements (such as level progression or animation) in order to teach or practice a particular topic\cite{GBLTeaching}. We find that GBL is been quite commonly used in the some fields of computer science, especially programming, however there are currently few examples of games that directly teach basic DSA. Those games that do exist to teach DSA usually revolve around more advanced DSA theory, such as algorithmic complexity. While learning those concepts are also important, there is a lack of games that help students learn about the practical application of DSA to computer science problems.\par
Another commonly studied educational tool is the concept of Algorithm Visualisation (AV). With AV, we try to abstract away the implementation of a given algorithm or data structure (in terms of, say, memory locations and pointers) and instead present them as static or dynamic diagrams or even analogies. Unfortunately while AVs are commonly seen in DSA teaching, most of these AVs are of the static, box-and-line diagram type\cite{Esponda-Arguero:2010:TVD:1827707.1827710}. When compared to dynamic visualisations that use analogies, these AVs may be effective at helping students correlate DSA with problems and implementations.\par
Therefore it is clear that there exists an opportunity for a educational tool that integrates both GBL and AV in ways that aren't commonly seen in the field of DSA.\par
DeCode is a game-based learning tool for teaching introductory data structures such as Arrays, Lists, Stacks, and Queues alongside basic algorithms that use said data structures. DeCode takes advantage of game-based level design and a 2.5D, animated depiction of data structures. The animation uses the metaphor of cars and parking spaces to better engage the student and help map key DSA concepts to a real-world analogy.\par
The research, implementation process and evaluation of DeCode will be discussed in this report, along with conclusions and possible future work.
\section{Related Work}
\subsection{Existing Algorithm Visualisations}
One of the first techniques used to improve instruction of DSAs was Algorithm Visualisations. While these were orignally limited to drawings on paper and blackboard, these eventually developed into software-generated interactive visualisations as early as 1984, with BALSA\cite{Brown:1984:SAA:964965.808596} as an early example of such. As development progressed, more user friendly visualisation tools were created, with most being freely distributed online as open source software, for use by educators around the world.\par
Despite this work however, the most common examples of AVs are still Static AVs (whether on a computer or paper), which do not allow the user to interact with the visualisation while using it, or allow analogies to be displayed. Recent research has shown that improved interactivity, explanations\cite{vegh2} and the ability for students to construct their own visualisations can help significantly improve the usefulness of AVs to students, especially those that struggle with traditional learning methods\cite{Stasko:1993:AAA:169059.169078}.\par
One of the most prominent algorithm visualisation tools avaliable now is VisuAlgo.net\cite{visualgo}, developed at the National University of Singapore. It offers a clean, modern, 2D visualisation of various data structures, and full interactivity for the user to perform operations on the data structure, and see what happens. The disadvantage of this approach is that the visualisations don't naturally lend to analogies to be constructed. We decided to implement the idea of having interactivity for the user to perform various operations on the data structure into our game, since we felt it allows for much better self-directed learning and experimentation.\par
While most AV tools we identified were designed purely for teaching and demonstration, the TRAKLA2\cite{TRAKLA2} tool was developed to be used as a student assessment tools. While it uses similar representations of DSA as VisuAlgo 
\begingroup

\section*{References}
  \vspace{2mm}

  % Delete the references header
  \renewcommand{\section}[2]{}

  % Reduce spacing
  \begin{spacing}{1.0}

    \bibliographystyle{IEEEtran}
    \small
    \bibliography{references}

  \end{spacing}

\endgroup
\end{document}