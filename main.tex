\documentclass[10pt]{article}
\usepackage{geometry}
\usepackage{setspace}
\usepackage{times}
\usepackage{titlesec}
\usepackage{enumitem}
\usepackage{graphicx}

\titleformat{\section}{\normalfont\bfseries\filcenter}{\thesection. }{0.8em}{}
\titlespacing*{\section}{0pt}{2.0ex}{-1ex}

\titleformat{\subsection}{\normalfont\bfseries}{\thesubsection. }{0em}{}
\titlespacing*{\subsection}{0pt}{2.0ex}{-1ex}

\titleformat{\subsubsection}{\normalfont\it}{\thesubsubsection. }{1.0em}{}
\titlespacing*{\subsubsection}{0pt}{2.0ex}{-1ex}

\geometry{
  a4paper,
  total={165mm,226mm},
  left=20mm,
  right=20mm,
  top=24.5mm,
  textwidth=165mm,
  textheight=235mm
}

\linespread{1.85}

\begin{document}

\noindent \textbf{ABSTRACT:} TODO!

\section{Introduction}
A solid understanding of Data Structures and Algorithms (DSA) is an important skill for any student in the field of Computer Science, since they underpin many of the fundamentals of Computer Science. However, as is the case with many conceptually demanding topics, students can find learning DSA challenging early on in their studies\cite{7600449} due to their inability to correlate DSA concepts with real-world objects and problems.\par
Educational tools have often been proposed that either take advantage of Game-based Learning (GBL) or Algorithm Visualisation (AV) in order to help students better learn concepts in computer science.\par
Game-based Learning is the use of games with traditional game elements (such as level progression or animation) in order to teach or practice a particular topic\cite{GBLTeaching}. We find that GBL is been quite commonly used in the some fields of computer science, especially programming, however there are currently few examples of games that directly teach basic DSA. Those games that do exist to teach DSA usually revolve around more advanced DSA theory, such as algorithmic complexity. While learning those concepts are also important, there is a lack of games that help students learn about the practical application of DSA to computer science problems.\par
Another commonly studied educational tool is the concept of Algorithm Visualisation (AV). With AV, we try to abstract away the implementation of a given algorithm or data structure (in terms of, say, memory locations and pointers) and instead present them as static or dynamic diagrams or even analogies. Unfortunately while AVs are commonly seen in DSA teaching, most of these AVs are of the static, box-and-line diagram type\cite{Esponda-Arguero:2010:TVD:1827707.1827710}. When compared to dynamic visualisations that use analogies, these AVs don't help students correlate DSA with real-world objects and problem.\par
Therefore it is clear that there exists an opportunity for a educational tool that integrates both GBL and AV in ways that aren't commonly seen in the field of DSA.\par
DeCode is a game-based learning tool 

\begingroup

  \section*{References}
  \vspace{2mm}

  % Delete the references header
  \renewcommand{\section}[2]{}

  % Reduce spacing
  \begin{spacing}{1.0}

    \bibliographystyle{IEEEtran}
    \small
    \bibliography{references}

  \end{spacing}

\endgroup
\end{document}